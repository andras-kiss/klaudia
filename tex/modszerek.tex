\chapter{Anyagok és módszerek}
\pagestyle{headings}

Az általam alkalmazott módszereket és az ehhez szükséges eszközöket, illetve anyagokat ismertetem ebben a fejezetben.

\section{Káliumion szelektív elektródok jellemzése}

Három kereskedelmi elektróddal dolgoztam, melyek közül kettő Thermo Scientific Orion 9719BNWP és egy WTW-Z0008I kombinált káliumion szelektív elektród.  

\subsection{Kalibrálás}
A mérések elvégzése előtt kalibráltam mindhárom elektródot. A következő lépések alapján végeztem a kalibrációt. KCl (Reanal Laborvegyszer Kft., Budapest, Magyarország) kristályvízmentes sójából bemértem analitikai mérlegen a 0.1 M koncentrációhoz szükséges előre kiszámolt tömeget, majd feloldottam kevés desztillált vízben, és egy 100 cm$^3$-es mérőlombikba öntöttem, ezután az oldatot kiegészítettem desztillált vízzel 100 cm$^3$ térfogatra, homogenizáltam az oldatot. Ezt követően 10 ml-t pipettáztam egy másik 100 ml-es mérőlombikba, majd jelretöltöttem vízzel, majd az új, tízszer hígabb oldatot homogenizáltam. A fenti lépéseket megismételtem, míg el nem értem a 10$^{-6}$ M koncentrációt. A só feloldásához, hígításhoz, valamint az edények mosási utáni átöblítéséhez használt desztillált vizet az Általános és Fizikai Kémia Tanszék vízdesztilláló készülékkel lett előállítva (Elix Essential 10 Water Purification System, központi cím), mely 0.067 $\upmu$S/cm fajlagos vezetésű nagytisztaságú víz előállítására képes. 

Az oldatkészítés befejezése után mindhárom elektródot kalibráltam úgy, hogy a leghígabb oldattól kezdve minden egyes oldatba bemerítettem az adott elektródot, és az egyensúlyi potenciál beállta után lejegyeztem a saját beépített referenciaelektródjához képest mért potenciált. A mérésekhez nagy bemeneti impedanciájú négycsatornás feszültségmérőt használtam (EPU452 Quad Multi Function isoPod™ with USB - eDAQ, 6 Doig Ave, Denistone East NSW 2112, Australia). A kapott potenciálértékeket grafikusan ábrázoltam a koncentráció negatív logaritmusának függvényáben. A pontokra egyenest illesztettem Qtiplot programmal.  

\subsection{Szelektivitás vizsgálat}

A szelektivitás adatokat a gyártó által publikált mérések alapján \cite{thermo} számoltam az \emph{azonos potenciálhoz tartozó aktivitásokból} különoldatos módszerrel \cite{buck1994recommendations}. A módszer az azonos potenciálhoz tartozó aktivitások arányaként adja meg a szelektivitási együtthatót, a \ref{eq:ssm2}. egyenlet alapján. Munkám során az aktivitások helyett mindenhol koncentrációkkal számoltam, ami esetemben elhanyagolható hibát okoz, tekintve, hogy híg oldatokkal dolgoztam. A káliumion kiáramlás vizsgálat során a legnagyobb tapasztalt koncentráció $\approx$10$^{-5}$ M volt.

\begin{equation}
k_{i,j}^{pot}=\frac{a_i}{a_j^{z_i/z_j}}
\label{eq:ssm2}
\end{equation}

\section{Sejttenyésztés}


\section{Extracelluláris káliumion koncentráció nyomonkövetése antifungális szerek jelenlétében}


