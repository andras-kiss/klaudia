\chapter{Anyagok és módszerek}
\pagestyle{headings}

Az általam alkalmazott módszereket és az ehhez szükséges eszközöket, illetve anyagokat ismertetem ebben a fejezetben.

\section{Káliumion szelektív elektródok jellemzése}

Három kereskedelmi elektróddal dolgoztam, melyek közül kettő Thermo Scientific Orion 9719BNWP és egy WTW-Z0008I kombinált káliumion szelektív elektród. Az elektródokat magam kalibráltam, a szelektivitási együtthatók számításához szükséges adatok a gyártótól származnak.

\subsection{Kalibrálás}
A mérések elvégzése előtt kalibráltam mindhárom elektródot. A következő lépések alapján végeztem a kalibrációt. KCl (Reanal Laborvegyszer Kft., Budapest, Magyarország) kristályvízmentes sójából bemértem analitikai mérlegen a 0.1 M koncentrációhoz szükséges előre kiszámolt tömeget, majd feloldottam kevés desztillált vízben, és egy 100 cm$^3$-es mérőlombikba öntöttem, ezután az oldatot kiegészítettem desztillált vízzel 100 cm$^3$ térfogatra, homogenizáltam az oldatot. Ezt követően 10 ml-t pipettáztam egy másik 100 ml-es mérőlombikba, majd jelretöltöttem vízzel, majd az új, tízszer hígabb oldatot homogenizáltam. A fenti lépéseket megismételtem, míg el nem értem a 10$^{-6}$ M koncentrációt. A só feloldásához, hígításhoz, valamint az edények mosási utáni átöblítéséhez használt desztillált vizet az Általános és Fizikai Kémia Tanszék vízdesztilláló készülékkel lett előállítva (Elix Essential 10 Water Purification System, központi cím), mely 0.067 $\upmu$S/cm fajlagos vezetésű nagytisztaságú víz előállítására képes. 

Az oldatkészítés befejezése után mindhárom elektródot kalibráltam úgy, hogy a leghígabb oldattól kezdve minden egyes oldatba bemerítettem az adott elektródot, és az egyensúlyi potenciál beállta után lejegyeztem a saját beépített referenciaelektródjához képest mért potenciált. A mérésekhez nagy bemeneti impedanciájú négycsatornás feszültségmérőt használtam (EPU452 Quad Multi Function isoPod™ with USB - eDAQ, 6 Doig Ave, Denistone East NSW 2112, Australia). A kapott potenciálértékeket grafikusan ábrázoltam a koncentráció negatív logaritmusának függvényáben. A pontokra egyenest illesztettem Qtiplot programmal.  

\subsection{Szelektivitás vizsgálat}

A szelektivitás adatokat a gyártó által publikált mérések alapján \cite{thermo} számoltam az \emph{azonos potenciálhoz tartozó aktivitásokból} különoldatos módszerrel \cite{buck1994recommendations}. A módszer az azonos potenciálhoz tartozó aktivitások arányaként adja meg a szelektivitási együtthatót, a \ref{eq:ssm2}. egyenlet alapján. Munkám során az aktivitások helyett mindenhol koncentrációkkal számoltam, ami esetemben elhanyagolható hibát okoz, tekintve, hogy híg oldatokkal dolgoztam. A káliumion kiáramlás vizsgálat során a legnagyobb tapasztalt koncentráció $\approx$10$^{-5}$ M volt.

\begin{equation}
k_{i,j}^{pot}=\frac{a_i}{a_j^{z_i/z_j}}
\label{eq:ssm2}
\end{equation}

\section{Sejttenyésztés}

A \emph{Candida albicans} sejttenyésztést rutin mikrobiológiai módszerekkel végeztem a PTE TTK Általános és Környezeti Mikrobiológiai Tanszéken Dr. Papp Gábor segítségével. Vizsgálataim során fontos volt a lehető legnagyobb sejtkoncentráció elérése, és az, hogy a sejttenyészet megfelelő, aktívan metabolizáló fázisban legyen. Ezek azért fontosak, hogy a lehető legnagyobb káliumion koncentráció változást tudjam detektálni, mely növeli a mérés pontosságát és lehetővé teszi az általánosan használt káliumion szelektív elektród használatát, melynek alsó kimutatási határa nagyjából 10$^{-6}$ M. A tenyésztés több napot vesz igénybe, és közben röbb mosási és átoltási lépés van, melyek között lehetőség van beállítani a következő tenyésztési lépés kiindulási sejtkoncentrációját. Ezt sejtszámolás helyett abszorbancia méréssel szokták végezni (hullámhossz).

A kiindulási \emph{Candida albicans} sejttenyészet mosásához nátrium-foszfát pufferoldatot használtam, ami izotóniás tulajdonsággal rendelkező sóoldat (PBS, Phosphate Buffered Saline 1X, pH 7.4, AAT Bioquest). Nem mérgező hatású a sejtekre nézve, így széles körben alkalmazható sejtek mosására vagy hígításokhoz, olyan készítményekre, melyek után visszamarad némi anyagfelesleg, de a szöveteket is ezzel kezelik, hogy könnyen lehessen szállítani őket. pH--értéke megközelíti az emberi testben mérhetőt, ugyanakkor hasonlóságot mutat az ozmolaritásban, ionkoncentrációban. Egyszerűen elkészíthető és jó az eltarthatósága. (nem kell ilyen gyártói reklámszöveg ide. ha esetleg nem te fogalmaztad meg, akkor saját szavakkal kéne újrafogalmazni)

A sejttenyészet előkészítésénél ügyelni kellett a steril környezet fenntartására, mert könnyen beszennyeződhet a vizsgálni kívánt sejttenyészet. Ezért a tenyésztéshez használt eszközöket sterilizáltam autoklávban vagy lánggal, és minden műveletet előzetesen sterilizált, mikrobiológiai elszívó fülke alatt végeztem. A kiindulási sejtszuszpenzió mért abszorbancia értéke 0.112 volt, ezt 48.8-szoros térfogatra hígítottam, majd spektrofotométerrel megmértem a hígított sejtszuszpenzió abszorbanciáját. Az elérni kívánt abszorbancia 0,0025 és a mért pedig 0,00222 volt, amit az együttműködő mikrobiológus kollégák elfogadhatónak találtak. Ezután szacharóz tartalmú tápanyagot pipettáztam mindkét lombikba, majd pedig az 1. számú lombikba 1 cm$^3$ hígítatlan (A=0,12) és a 2. számú lombikba 2 cm$^3$ hígított szuszpenziót (A=0,222). Rázógépbe helyeztem el a lombikokat, ahol 30 °C-os hőmérsékleten 13 órán át (ide azért kéne egy ige, pl "volt" vagy "növekedett"), ez idő alatt nőtt a sejtszám a szuszpenzióban. az ülepedés megakadályozása érdekében folyamatos volt a lombikok rázása. 

A méréshez szükséges kiindulási sejtszuszpenziót spektrofotométerrel (egyfényutas, UV/VIS tartomány (korábban is ezt használtad, az első előfordulási helyen kell leírni)) mértem meg, a mért abszorbancia érték 0,459 volt. A szükséges koncentráció elméleti abszorbancia értéke 0,015, ehhez 30,6-szoros hígításra volt szükség, viszont figyelembe kellett venni, hogy előzőleg már elvégeztem egy 10-szeres hígítást, ezért a végleges 306-szoros volt. Ezt követően szacharóz tartalmú tápanyagot pipettáztam mindkét lombikba, majd az 1. számú lombikba 1 cm$^3$ hígítatlan (A=0,459) és a 2. számú lombikba 2 cm$^3$ hígított szuszpenziót (A=0,015).  Rázógépbe helyeztem el mindkét lombikot, ahol 29,5°C-os hőmérsékleten 150 rpm (round per minute) keverési sebesség mellett 18 órán át folyamatos rázás alatt állt.

\section{Extracelluláris káliumion koncentráció nyomonkövetése antifungális szerek jelenlétében}

Az előző nap folyamán a \emph{Candida albicans} sejttenyészetből készített szuszpenziót háromfele osztottam. Két 10 cm$^3$ térfogatú főzőpohárba 6-6 cm$^3$ -t és egy 10 cm$^3$ térfogatú főzőpohárba 8 ml-t pipettáztam. A két 6 cm$^3$ térfogatú szuszpenziókhoz adott időpontban antifungális szereket adtam meghatározott térfogatban (30 $\upmu$l), a harmadik oldatot kezeletlenül hagytam. Mindhárom oldatot kevertettem a mérés során, hogy megakadályozzam a sejtek a főzőpohár aljára történő ülepedését. A 2. csatornán Orion elektródot helyeztem a propolisszal kezelt szuszpenzióba és mértem az elektród potenciálváltozását. A 3. csatornára volt csatlakoztatva a WTW elektród, ezt a kezeletlen szuszpenzióba tettem (negatív kontrol). A 4. csatornán szintén Orion elektród merült az szuszpenzióba, ezt nysztatinnal kezeltem. Adott időpontban adtam hozzá az antifungális szereket mindkét szuszpenzióhoz (17 min-nél propolisz, 17.6 min-nél nysztatin). NaOH oldatot (0,6 cm$^3$ és 0,8 cm$^3$) oldatot adagoltam a nysztatinnal kezelt mintához, itt nem történt változás, viszont a propolisszal kezelt mintához adva feszültségnövekedés volt megfigyelhető, ez azzal magyarázható, hogy erős lúg hatására a még szét nem esett sejtek is elpusztultak. 

Az mérés időpontját megelőző napon elkészített \emph{Candida albicans} gombatenyészetből (10$^7$ db sejt/ml, A=0,625) készített szuszpenziót háromfelé osztottam. Két 10 cm$^3$ térfogatú főzőpohárba 5-5 cm$^3$-t és egy 10 cm$^3$ térfogatú főzőpohárba 3 cm$^3$-t pipettáztam. A két 5 cm$^3$ térfogatú szuszpenziót adott időpontban meghatározott térfogatú antifungális szerekkel kezeltem. A harmadik szuszpenziót kezeletlenül hagytam. Mindhárom oldatot kevertettem a mérés során, hogy a sejtek ne ülepedjenek le a főzőpohár aljára. A 2. csatornán Orion elektróddal a propolisszal kezelt szuszpenzióban mértem az elektród potenciálváltozását. A 3. csatornára volt csatlakoztatva a WTW elektród, ez a kezeletlen szuszpenzióba merült (negatív kontrol). A 4. csatornán szintén Orion elektród követtem nyomon a változást a szuszpenzióban, ezt nysztatinnal kezeltem. Adott időpontban adtam hozzá az antifungális szereket mindkét szuszpenzióhoz (15 min-nél 50 $\upmu$l propolisz, 15.3 min-nél 50 $\upmu$l nysztatin). Ezt követően 2 óra elteltével mindhárom szuszpenzióhoz térfogatarányosan adagoltam nysztatint. Ezzel a lépéssel kizárhattam azt, hogy esetlegesen az elektródból kiszivárgott elektrolit oldat okoz koncentrációváltozást vagy, hogy a hatóanyag okozott sejtszétesést, ami szintén növeli az extracelluláris kálium-ionkoncentrációt. Túl kicsi sejtszám esetén nem biztos, hogy detektálni tudom a változást az elektróddal. Akár megdőlhet a főzőpohár és kifolyik a szuszpenzió egy része, így megint nem a tényleges koncentrációt detektálom, hanem a ténylegesnél kisebb értéket. 

Mindkét mérés során online módszert alkalmaztam, vagyis, hogy időben folyamatos volt a változás nyomon követése, így a sampling módszerrel szemben a legrövidebb idő alatt végbemenő változást is detektálni tudtam, nemcsak a kiindulási-és a végállapotot. A mérési adatokból pedig grafikusan tudtam ábrázolni, hogy az idő tört része alatt hogyan változott a potenciál az egyensúly beálltáig. A rendelkezésre álló kalibrációs egyenes egyenletének segítségével a potenciál értékekből pedig koncentrációt számoltam.  


