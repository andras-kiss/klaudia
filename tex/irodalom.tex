\chapter{Irodalom}
\pagestyle{headings}

Szakdolgozatomban interdiszciplináris témát dolgozok fel, mivel a vizsgálat tárgya (Candida albicans) mikrobiológiai területhez tartozik, illetve az elektrokémiában és elektroanalitikában alkalmazott eszközöket és módszereket használtam fel a munkám során.
Ebben a fejezetben tudománytörténeti visszatekintést teszek annak céljából, hogy viszonylag röviden összefoglalom az ionszelektív elektródok hogyan váltak az analitikai vizsgálatok általános eszközévé. Kifejlesztése összekapcsolódik a biológiával és az orvostudománnyal, hiszen Max Cremer fiziológusként szerette volna megvalósítani, hogy a sejtmembránon belül és kívül bekövetkező fiziológiai változásokat valós időben követni tudja, viszont az ehhez szükséges eszköz  azonban még nem állt rendelkezésére. Felfedezte, hogy az üveg a szemipermeábilis hártyához hasonlóan csak adott ionok számára biztosít szabad vándorlást, ezen elv alapján megalkotott egy elektródot, aminek az érzékelő része üvegmembránból készült. A membrán szabadon átjárható a H+-ionok számára, ezáltal pH mérésre is képes volt. Tulajdonképpen az üvegelektród is egy ionszelektív elektródnak tekinthető. Cremer a sejtek és az elektrokémiai cellák között analógiát fedezett fel, míg a cellákat egyfázisú láncoknak tekintette, addig a sejteket többfázisú láncoknak. Rendelkezésére álltak az elméleti elektrokémia területén szerzett legfrissebb újdonságok és változások, melyeknek előre gondosan megtervezett kísérleti kutatások akarták bizonyítani valódiságát.\cite@article{scholz2011leiden,
  title={From the Leiden jar to the discovery of the glass electrode by Max Cremer},
  author={Scholz, Fritz},
  journal={Journal of Solid State Electrochemistry},
  volume={15},
  number={1},
  pages={5--14},
  year={2011},
  publisher={Springer}
}
 Ismerte W.Nernst-t, aki a XX. század egyik legmeghatározóbb tudósai közé tartozott a fizikai kémia területén. A sajtó, illetve a tudományos közösség mindig a triumvirátus (Ostwald, van’t Hoff, Arrhenius) leszármazottakként emlegette. \cite@book{barkan1999walther,
  title={Walther Nernst and the transition to modern physical science},
  author={Barkan, Diana Kormos and Buchwald, Diana Kormos},
  year={1999},
  publisher={Cambridge University Press}
}
 Később más kutatók is foglalkoztak az üvegelektród fejlesztésével, új módszerekkel próbálkoztak, melyek eredményei megfeleltek az eddigi elméleti feltevéseknek vagy pedig módosították és az elmélet újragondolását és átalakítását kívánták meg. A membrán anyagát képező üveget Klemensiewicz és Haber további eljárások útján specifikusabbá és ellenállóvá tették.\cite@article{scholz2011leiden,
  title={From the Leiden jar to the discovery of the glass electrode by Max Cremer},
  author={Scholz, Fritz},
  journal={Journal of Solid State Electrochemistry},
  volume={15},
  number={1},
  pages={5--14},
  year={2011},
  publisher={Springer} Az ionszelektív üvegelektród felületén végzett koncentrációprofilokról elsőként Boksay számolt be, aki a nátrium és kálium eloszlását HF-os frakcionált oldással határozta meg. Ezt követően számos mikroanalitikai módszert alkalmaztak elektródfelületek, elektrokémiai filmek  tulajdonságainak elemzésére, elektrokémiai problémák megoldására, majd ezeket megpróbálták adaptálni ion-szelektív elektródok membránfelületeit vizsgáló eljárásokra. Ionporlasztást és infravörös spektroszkópiát üvegmembrán korróziójának vizsgálatához, röntgenfluoreszcencia analízist, pásztázó elektronmikroszkópiát és röntgenvizsgálatot alkalmaztak a korróziós folyamatok megfigyeléséhez AgI/Ag2S-ot tartalmazó membránokban.\cite@article{pungor1990general,
  title={General survey and microanalytical aspects of ion-selective electrodes},
  author={Pungor, Ern{\H{o}} and Lindner, Ern{\H{o}} and T{\'o}th, Kl{\'a}ra},
  journal={Fresenius' Journal of Analytical Chemistry},
  volume={337},
  number={5},
  pages={503--507},
  year={1990},
  publisher={Springer}
} Eisenman szintén a membrán típusú felépítést elvét követte, azonban folyadék halmazállapotú anyagot használt üveg helyett, ez pedig a polivinil-klorid(PVC) volt. 1966-ban publikált kötetében az üvegelektródot egy permszelektív, homogén ioncserélőnek tekinti.\cite@article{conti1965steady,
  title={The steady-state properties of ion exchange membranes with fixed sites},
  author={Conti, Franco and Eisenman, George},
  journal={Biophysical journal},
  volume={5},
  number={4},
  pages={511--530},
  year={1965},
  publisher={Elsevier}
}
\cite@article{sandblom1967electrical,
  title={Electrical phenomena associated with the transport of ions and ion pairs in liquid ion-exchange membranes. I. Zero current properties},
  author={Sandblom, John P and Eisenman, George and Walker Jr, John Lawrence},
  journal={The Journal of physical chemistry},
  volume={71},
  number={12},
  pages={3862--3870},
  year={1967},
  publisher={ACS Publications}
}
 Simon és Walker Eisenmanhoz hasonlóan élő szervezetek viselkedésének tanulmányozásához ionszelektív elektródot használt. Az évek során az elektród szelektivitása egyre több ionra kiterjedt, ezáltal a zavaró ionok jelentléte kevésbé jelentett problémát. Ionofór vegyületek megjelenésével még specifikusabban lehetett a mintát egy adott ionra vonatkozóan vizsgálni. 
 
 
\sectionIonszelektív membrán elektródok
\subsectionIonofórt tartalmazó ionszelektív elektródok történeti háttere 
A mai, kereskedelmi forgalomban kapható ionszelektív elektródok (ISE) történeti háttere egészen az 1960-as évekig nyúlik vissza, amikor a már fentebb említett üvegelektród (ami hidrogénionra nézve szelektív) helyett egy hordozó alapú ISE kifejlesztése volt a cél. Egymástól független kutatások viszonylag hamar hoztak átütő sikereket. Sejtek antibiotikumokkal történő kezelése során megfigyelték, hogy az antibiotikumok közül némelyik képes iontranszportot indukálni a mitokondriumban. Míg más vegyületek esetében in vivo módon történt a szelektivitás, az aintibiotikumok esetén ez in vitro módon történt.  Évekkel később (Simon, Stefanac) pedig sikerült megtalálni rá a magyarázatot, ami szerint a komplex formációjából fakadóan adott ionra szelektív. Pedersen és Lehn is megfigyelte, hogy alkáli-és alkáliföldfém-ionokkal a mesterségesen előállított makrociklusos poliéterek és makroheterobiciklusos vegyületek komplexeket képeznek.  Az ezt követő években számos tanulmány és értekezés született a természetes és mesterséges ionofórokkal és az általuk fémekkel alkotott komplexeikkel kapcsolatosan, ami megint egy fontos mérföldkőnek számított ezen a területen.  Az első kalcium-ionra szelektív ISE membránjában elektromosan semleges volt az ionofór vegyület és töltés nélküli hordozókat tartalmazott. Azonban nem mindegyik ionofór vegyület rendelkezett makrociklusos szerkezettel, ami megcáfolta azt a korábbi elképzelést, hogy ha egy vegyületet komplexképzőként lehet jellemezni, akkor biztosan makrociklusos felépítésű. Az akkor ismert korona-éterek és kriptandok nem bizonyultak megfelelőnek, hogy ionofórként használják az elektród membránban. Elsősorban azért, mert az éterek alacsony lipofilitással és korlátozott szelektivitással bírtak, a kriptandok pedig nem voltak lipofil tulajdonságúak és lassan következik be az ő esetükben komplexképzés, viszont nagy a szelektivitásuk.  A kémia azon részén folytatott kutatások, amelyek a szupramolekuláris és a host-guest kölcsönhatásokkal és az ahhoz szükséges körülmények megteremtésével foglalkoztak, elősegítették a makrociklusos vegyületek kémiájának megértését és fejlődését. Szintén fontos előre lépésnek számított, mikor Shatkay és munkatársai, ugyanakkor Ross is az oldószeres polimer alapú membránokat épített be az ISE-ba. Napjainkban is a polivinil-klorid (PVC) alapú membrán terjedt el a mindennapi használtban, viszont emellett más polimer mátrixokat is alkalmaznak a gyakorlatban. \cite@article{bakker1997carrier,
  title={Carrier-based ion-selective electrodes and bulk optodes. 1. General characteristics},
  author={Bakker, Eric and B{\"u}hlmann, Philippe and Pretsch, Ern{\"o}},
  journal={Chemical Reviews},
  volume={97},
  number={8},
  pages={3083--3132},
  year={1997},
  publisher={ACS Publications}
}

\subsectionCsapadék alapú elektród
Az ionszelektív membránok egyik csoportját képezik a csapadék alapú elektródok, ahol  lényegében az elektródaktív anyag maga a csapadék alapú membrán, melynek  az oldószerben való oldhatósága meghatározó az elektród további tulajdonságait illetően, szemben más elektródokkal.\cite@incollection{pungor1973precipitate,
  title={Precipitate-based ion-selective electrodes},
  author={Pungor, E and Toth, K},
  booktitle={Analytical Chemistry--4},
  pages={105--137},
  year={1973},
  publisher={Elsevier}
}
 Előállításukra több módszer közül lehet választani. Az első szerint egyetlen kristályból elkészíthető a membrán, a második módszernél már préselt kristályokat lehet felhasználni membrán alapanyagként, a  harmadik  metódus pedig kristályokat ágyaz a megfelelő anyagba. Az első két módszer szerint elkészített készülékek a homogén elektródok, míg a harmadik típus a heterogén elektródok csoportjába sorolható. E két elektródtípus között azonban nincs alapvető különbség. Természetesen a mechanikai tulajdonságokat, élettartamot tekintve előfordulnak, de elektrokémiai szempontból ezeket együtt kell kezelni. Viszont nagyon fontos tényező a kristályszerkezet és a benne esetlegesen előforduló rácshibák, melyek befolyásolják az analitikai paramétereket, mint pl: az elektród standard potenciálja, hiszen a membrán anyagát képezi maga a kristály.\cite@article{pungor1990general,
  title={General survey and microanalytical aspects of ion-selective electrodes},
  author={Pungor, Ern{\H{o}} and Lindner, Ern{\H{o}} and T{\'o}th, Kl{\'a}ra},
  journal={Fresenius' Journal of Analytical Chemistry},
  volume={337},
  number={5},
  pages={503--507},
  year={1990},
  publisher={Springer}
}
\cite@incollection{pungor1973precipitate,
  title={Precipitate-based ion-selective electrodes},
  author={Pungor, E and Toth, K},
  booktitle={Analytical Chemistry--4},
  pages={105--137},
  year={1973},
  publisher={Elsevier}
}
Kondícionálóval vagy pedig az elektród anódos és katódos polarizációjával növelhető az érzékenység, alacsony koncentrációtartományokban ajánlott elvégezni a kondícionálást, mert így állandó kezdeti körülményeket lehet teremteni.  Kimutatási határ esetén az oldhatósági szorzatot tekintjük mérvadónak.\cite@incollection{pungor1973precipitate,
  title={Precipitate-based ion-selective electrodes},
  author={Pungor, E and Toth, K},
  booktitle={Analytical Chemistry--4},
  pages={105--137},
  year={1973},
  publisher={Elsevier}
}
\cite@article{pungor1990general,
  title={General survey and microanalytical aspects of ion-selective electrodes},
  author={Pungor, Ern{\H{o}} and Lindner, Ern{\H{o}} and T{\'o}th, Kl{\'a}ra},
  journal={Fresenius' Journal of Analytical Chemistry},
  volume={337},
  number={5},
  pages={503--507},
  year={1990},
  publisher={Springer}
}A határfelületi potenciálkülönbségeket a fő ionok ioncseréje hozza létre. A határfelületi potenciált azok a primer ionok aktivitása határozza meg, melyek az oldatban érintkeznek a membrán felületével és a membránban megtaláható primer ionok pedig azok, melyek a membrán azon felületén helyezkednek el, ami az oldattal érintkezik. Minden olyan folyamat, ami változást idéz elő a membrán primer ionösszetételében vagy megváltoztatja az oldat aktivitását, hatással van a határfelületi potenciálra. A fizikai adszorpció azonban nem befolyásolja a határfelületi potenciált, ha a fázisösszetétel állandó marad.\cite@article{berube1989comparison,
  title={Comparison of proposed response mechanisms of precipitate-based ion-selective electrodes in the presence of interfering ions},
  author={Berube, Thomas R and Buck, Richard P and Lindner, Erno and Gratzl, Miklos and Pungor, Erno},
  journal={Analytical Chemistry},
  volume={61},
  number={5},
  pages={453--458},
  year={1989},
  publisher={ACS Publications}
}


\subsectionFolyadékmembrán elektród
A vizsgálati oldatban fel nem oldódó oldószerréteg alkotja a membránt. Az Orion kutatói folyékony ioncserélővel telített rugalmas, porózus műanyag membránt használtak fel az elektródokhoz, a Corning Glass Laboratories pedig zsugorított üvegszűrőt. Próbálkoztak már számos anyaggal, ami alkalmas lehet a folyadék vizsgálandó anyagtól történő elkülönítésére. Tulajdonságait tekintve a legfontosabb, hogy vízben oldhatatlan, alacsony gőznyomás jellemzi, nem tartalmazza a membránt telítő folyadékot az oldószer. A membrán stabilitását meghatározza a folyadék viszkozitása. Nagy molekulatömegű, kis dielektromos állandójú folyadék esetén, nagymértékű iontársulás fog bekövetkezni, ami vizsgálandó ionra nézve nagy szelektivitást mutat. Számos tényező, többek között az oldószer is befolyásolja az ion által képzett komplex stabilitását. Sollner, Sandblom, Orme a különböző folyadékmembrán rendszereket és a feltételezett transzportjelenségeket kutatta, vizes fázisban a membrán felületéhez közel lévő szabad ionok között alakult ki, majd ugyanezek az ionok kötődnek a felületi szerves csoportokhoz. Az ioncsere folyamatának szelektivitása határozza meg elsősorban az elektród szelektivitását. Sokféle folyékony ioncserélőt kipróbáltak annak érdekében, hogy egy használható elektródot tudjanak előállítani. Előnyben részesülnek azok a kation- és anioncserélők, melyek ellentétes töltéseket hordoznak, viszont rossz szelektivitás mutatnak bármely kationnal vagy anionnal szemben. Ionpreferenciájukat vizsgálva az tapasztalható, hogy hasonlóságot mutatnak a Hofmeister liofil sorozattal. Egyes elektroneutrális makrociklusos vegyületek esetében bizonyos kationokkal szemben jó szelektivitás figyelhető meg. 

\subsectionKálium-szelektív membrán elektród
Nagy szelektivitású elektródákban a membrán alkotórészeként alkalmaznak makrociklusos vegyületeket, mert képesek megkülönböztetni egymástól a hasonló tulajdonságokkal rendelkező ionokat. Kezdetben a nonactin és homológjai töltötték be az ionofór vegyület szerepét, viszont az Orion Research és a Philips kifejlesztett egy olyan membrán elektródot, amiben egy Millippore szűrő volt beépítve, ez pedig telített valinomicin oldatot tartalmazott. Analitikai szempontból a valinomicin egy nagyon alkalmas vegyület, ha K+-iontartalmú mintát kell vizsgálni, mert 5000-szer nagyobb szelektivitást mutat a Na+-ionhoz képest. 10-1-10-6 M kálium-ion koncentrációtartományban az elektródválasz megfelel a Nernst- egyenletnek:

E a cellapotenciál, E0 a rendszer standard potenciálja, R az egyetemes gázállandó, T a termodinamikai hőmérséklet, z az elektródfolyamat elektronszámváltozása, F a Faraday- állandó, a az elektródaktív komponens aktivitása.(\cite@book{lakshminarayanaiah2012membrane,
  title={Membrane electrodes},
  author={Lakshminarayanaiah, Nallanna},
  year={2012},
  publisher={Elsevier}
} Az ionok egyensúlyi koncentrációját és a passzív membrán viselkedését leíró elméleti feltételezéseket kapcsolja össze a potenciálgáton keresztül a Nernst-egyenlet, mely a Goldman- egyenlethez hasonlóan a hőmérséklettől függ. A hibák elkerülése végett ezért állandó körülmények között kell elvégezni a membrán komponenseinek a vizsgálatát. \cite@article{barnes1984cell,
  title={Cell membrane temperature rate sensitivity predicted from the Nernst equation},
  author={Barnes, Frank S},
  journal={Bioelectromagnetics: Journal of the Bioelectromagnetics Society, The Society for Physical Regulation in Biology and Medicine, The European Bioelectromagnetics Association},
  volume={5},
  number={1},
  pages={113--115},
  year={1984},
  publisher={Wiley Online Library} 
  } A tömeges kísérleti bizonyítékok lehetővé tették, hogy a Nernst- egyenlet a gyakorlatban is alkalmazhatóvá váljon, viszont figyelembe kell venni, hogy az egyenlet klasszikus levezetése az elektrokémiai potenciálon alapul, ami a kémiai potenciál és az egységnyi területre eső moláris töltés összegeként írható fel. Közvetlenül származtatható alapvető termodinamikai elvekből, így a Boltzmann- eloszlásból és a tér-töltés  között kialakult kölcsönhatási energiából.  \cite@article{feiner1994nernst,
  title={The nernst equation},
  author={Feiner, A-S and McEvoy, AJ},
  journal={Journal of chemical education},
  volume={71},
  number={6},
  pages={493},
  year={1994},
  publisher={ACS Publications}
}
Viszont mások arra jutottak, hogy 10-1-10-4 és 10-1-10-5 M-os koncentrációk esetén a görbe meredeksége 58,3mV/tized. Lal ás Christian megállapította, hogy K/Na szelektivitási arány függ a két ion relatív koncentrációjától. Jelentősen befolyásolja az elektródválaszt jodid-, hidroxid-, kromát-, oxalát-ion jelenléte. Az akkori elektród felhasználási területe még korlátok közé volt szorítva, amit megerősít, hogy tetrafenil-borát jelenléte jelentős potenciáleltolódásokat okozott. Hammond és Lambert ehhez hasonlóan megvizsgálták szerves és szervetlen ionokra nézve, hogy milyen elektródválaszokat kapnak, illetve hogy a membránban oldható tetrafenil-borát interferenciát okoz-e az elektródválaszban. A tetrafenil-metil borát mellett interferenciát okozott a pozitív töltésű, felületaktív tulajdonsággal rendelkező cetil-trimetil-ammónium-bromid, de a hozzá hasonlóan felületaktív, negatív töltésű nátrium-dodecil-szulfát és a nem felületaktív, pozitív töltésű tertametil-ammónium-bromid nem okozott interferenciát. Ezekből az eredményekből arra lehet következtetni, hogy csak akkor jelentkezik interferencia az elektródválaszban, ha nettó pozitív töltésű és felületaktív tulajdonságokkal rendelkező vegyület van jelen a vizsgált mintában/oldatban. A káliumion oldatokban történő közvetlen kimutatására valinomicin alapú K+-ionszelektív elektródot használtak, különböző módszerek alapján. \cite@book{lakshminarayanaiah2012membrane,
  title={Membrane electrodes},
  author={Lakshminarayanaiah, Nallanna},
  year={2012},
  publisher={Elsevier}
} 
\subsectionSzelektivitás elméleti alapjai
Az ionszelektív elektród egyik legfontosabb jellemzője, amiről nem szabad megfeledkezni az a szelektivitás, hiszen a kimutatási határ mellett ez is meghatározza, hogy a mintában történő mérés későbbi eredményei megbízhatónak bizonyulnak-e. Klinikai vizsgálatok esetén ez különösen fontos, hiszen ott a maximális eltérés nem lehet nagyobb 0.1mV-nál.  A szelektivitás elméleti részletes leírása lehetővé teszi a kutatók számára, hogy a legfontosabb paramétereket (abszolút membránkoncentráció, a használni kívánt lágyító és mátrix) figyelembe véve tudják növelni a potenciometriás érzékelők optimális teljesítményét. Gyakorlatilag a szelektivitási szempontok a Nikolsky- Eisenman egyenleten alapulnak, azonban ennek az egyenletnek is vannak korlátai és némileg pontatlan, fő hibája, hogy ha a zavaró és az elsődleges ion töltése között eltérés áll fent, akkor szelektivitási tényezőt nem lehet egyértelműen meghatározni. Azonban ha felcserélnénk a két iont, akkor eltérő lenne a kapott elektródválasz is.\cite@article{erdHosy2001planaris,
  title={Plan{\'a}ris fel{\'e}p{\'\i}t{\'e}s{\H{u}} ionszelekt{\'\i}v elektr{\'o}dok elm{\'e}lete {\'e}s biol{\'o}giai alkalmaz{\'a}sa},
  author={Erd{\H{o}}sy, Mikl{\'o}s},
  year={2001},
  publisher={Budapesti M{\H{u}}szaki {\'e}s Gazdas{\'a}gtudom{\'a}nyi Egyetem}
}
A fázishatáron kialakuló potenciált leíró modellből származtatták azt levezetést, amely megmagyarázza az oldószeres polimer membrán ISE által adott vegyes ionválaszt. Ideális körülmények között az elektródválasz függvénye követi a Nernst-egyenletet. Minden ion rá jellemző potenciál állandóval rendelkezik, amit a Nernst-egyenlet E0I formában tüntet fel.  A Nikolsky- Eisenman formalizmus szerint az egyenletben szereplő aktivitási tényezőt kicserélhetjük a mintában szereplő ionok aktivitásainak az összegével, ha a mintaoldat több iont tartalmaz. Rendkívül szelektív elektród esetében nagyon kis értékű lesz a Nicolsky-féle szelektivitási tényező és az oldatban lévő ionok aktivitása közel azonos. Interferencia akkor következik be, ha a kevert mintában lévő alacsony aktivitással rendelkező ion ugyanazt az elektródválaszt adja, mint az oldószerben található, ismert aktivitású ion és emellett tudjuk, hogy az oldószer nem tartalmaz zavaró ionokat.\cite@article{bakker1997carrier,
  title={Carrier-based ion-selective electrodes and bulk optodes. 1. General characteristics},
  author={Bakker, Eric and B{\"u}hlmann, Philippe and Pretsch, Ern{\"o}},
  journal={Chemical Reviews},
  volume={97},
  number={8},
  pages={3083--3132},
  year={1997},
  publisher={ACS Publications}
}


A szelektivitási állandó meghatározásának főbb gyakorlati módszerei
Nikolszky-féle szelektivitási tényezőt kétféle módon lehet meghatározni. A különoldatos módszer lényege, hogy két egymástól elválasztott oldat csak az elsődleges ion sóját és a zavaró iont tartalmazza.  Ha mindkét oldatban azonos elektromotoros erőt mérünk, akkor az alábbi egyenlet lesz érvényben:(11. egyenlet)

A másik módszer lényege, hogy az interferencia állandó, ennek az lesz a következménye, hogy az elsődleges ionhoz tartozó kalibrációs görbét a háttérelektrolitban lévő zavaró ionok fogják befolyásolni és meghatározni. A kalibrációs görbén belül két egymástól jól megkülönböztethető Nernst-szakasz figyelhető meg, ezek olyan tartományokra vonatkoznak, melyekben egyetlen ion határozza meg a potenciált. A tartományok segítségével meghatározható a Nikolszky-féle szelektivitási tényező \cite@article{bakker1997carrier,
  title={Carrier-based ion-selective electrodes and bulk optodes. 1. General characteristics},
  author={Bakker, Eric and B{\"u}hlmann, Philippe and Pretsch, Ern{\"o}},
  journal={Chemical Reviews},
  volume={97},
  number={8},
  pages={3083--3132},
  year={1997},
  publisher={ACS Publications}
}

\subsectionÉrzékenység
Kereskedelmi forgalomban kapható ISE által adott elektródválasz megfeleltethető a Nernst-egyenletnek. A mérés során megfigyelhető az elektród érzékenységének csökkenése, miközben a vizsgált ion z*I töltése folyamatosan növekszik, ezen megfigyelés alapján megállapítható, hogy az érzékenység felírható a mért jel és a vizsgált ion töltésének hányadosaként. 10-szeres aktivitásváltozás esetén a mért elektromotoros erő (EMF) 59mV, ha az vizsgált ion egyszeres töltésű, kétszeres töltéssel rendelkező ion aktivitásváltozásánál 58mV/dekád a mérhető potenciálváltozás, ez kétszeres Nernsti válasz (Bahlman et al.). Azonban a kétszeres Nersnti válasz látszólagos tartományán belül történő ioncsere a főbb membránalkotók koncentrációját nem befolyásolja. Az ilyenkor kapott elektródválasz sokkal gyorsabb, mint az egyszeres töltésű ionok esetén kapott Nernsti válasz. \cite@article{bakker2000ionophore,
  title={Ionophore-based membrane electrodes: new analytical concepts and non-classical response mechanisms},
  author={Bakker, Eric and Meyerhoff, Mark E},
  journal={Analytica chimica acta},
  volume={416},
  number={2},
  pages={121--137},
  year={2000},
  publisher={Elsevier}
}

\subsectionAz ionszelektív elektródok kimutatási határa
A többi analitikai eszközhöz hasonlóan az ionszelektív elektródok is jellemezhetők alsó és felső kimutatási határral. A két határ által közrefogott lineáris tartományon kívülre eső szakaszokon az elsődleges ionnal szembeni Nernst szerinti érzékenységét elveszíti az elektród. \cite@article{erdHosy2001planaris,
  title={Plan{\'a}ris fel{\'e}p{\'\i}t{\'e}s{\H{u}} ionszelekt{\'\i}v elektr{\'o}dok elm{\'e}lete {\'e}s biol{\'o}giai alkalmaz{\'a}sa},
  author={Erd{\H{o}}sy, Mikl{\'o}s},
  year={2001},
  publisher={Budapesti M{\H{u}}szaki {\'e}s Gazdas{\'a}gtudom{\'a}nyi Egyetem}
} Definíció szerint az a koncentráció, amely hatására adott körülmények között a potenciál(E) egyre jobban távolodik az I (lineáris) tartományban mérhető átlagos potenciáltól. (fig1) A rendelkezésre álló technikai lehetőségek és az elvárás, hogy gyakorlati szempontból kényelmesebb és egyszerűbb legyen a kimutatási határ megállapítása, ezért azt az aktivást(koncentrációt) tekintik az alsó értéknek, ami kalibrációs görbe extrapolált lineáris szakaszaira illesztett egyenesek metszéspontjánál olvasható le. Számos tényezőt figyelembe kell venni a kimutatási határ meghatározásánál, ezek közül a fontosabbak: az oldatösszetétel, a használni kívánt elektród tulajdonságainak és a működéshez szükséges körülmények ismerete. \cite@article{ionrecommendations,
  title={Recommendations for Nomenclature of ION-Selective Electrodes},
  author={ION-SELECTIVE, NOMENCLATURE OF}
}
 Napjainkban a IUPAC 1976-os ajánlása van érvényben, de voltak más javaslatok is a kimutatási határ definíciójára. 
Az alsó kimutatási határ jelenlétének egyik magyarázata, hogy ha az elsődleges ion alacsony koncentrációban van jelen az oldatban, hiszen a membránból történő kiáramlása folyamatos, ennek pedig az lesz a következménye, hogy a határfelületen nullánál nagyobb lesz az elsődleges ionkoncentráció és a határfelületi ionaktivitás a membránban mérhetőhöz képest. Az elektródpotenciál független lesz a mintaoldatban lévő ionaktivitástól, ekkor a Nernst-féle elektródválasz megszűnik. A másik magyarázat pedig az hogy a zavaró ion(ok) interferáló hatást okoznak. Normális körülmények esetén az alsó kimutatási határ 10-6mol/dm3, de ez nagyságrendekkel csökkenthető elsődleges ionpufferek alkalmazásával.
A felső kimutatási határ létezését szintén bizonyították a permszelektivitás csökkenésének jelenségével, ami annak a következménye, hogy az elsődleges és az interferenciát előidéző ion extrahál a mintaoldatból a membránfázisba. A jelenség, közismert nevén a Donnan-féle hiba leginkább az ionofór alapú kationszelektív membránoknál fordul elő. 
\cite@article{erdHosy2001planaris,
  title={Plan{\'a}ris fel{\'e}p{\'\i}t{\'e}s{\H{u}} ionszelekt{\'\i}v elektr{\'o}dok elm{\'e}lete {\'e}s biol{\'o}giai alkalmaz{\'a}sa},
  author={Erd{\H{o}}sy, Mikl{\'o}s},
  year={2001},
  publisher={Budapesti M{\H{u}}szaki {\'e}s Gazdas{\'a}gtudom{\'a}nyi Egyetem}
}

\subsectionIonszelektív elektródok válaszideje
Azok az analitikai szempontból fontos jellemzők (élettartam, alsó kimutatási határ, szelektivitási faktor, válaszidő), melyekkel az ionszelektív elektródokat adott típusú feladatok elvégzése szempontjai szerint lehessen osztályozni, ugyanakkor ezek állandóságától függ, hogy az elektród meddig képes hiteles mérési eredményekkel szolgálni. Meghatározásukhoz a kísérleti módszereket gondosan kell megválasztani, hiszen így lehetőség nyílik arra, hogy az egymástól ISE-k teljesítményjellemzőit össze tudjuk hasonlítani. Különösen fontos a válasz idő meghatározásánál a legideálisabb mérési technika megválasztása és hogy a kísérleti paraméterek pontosítva legyenek, mert  az adott típusba tartozó készülékek esetén is nagyságrendekkel eltérhet a meghatározott érték. Az ionaktivitás segítségével is meg lehet határozni a tényleges válaszidőt. A módszer lényege, hogy a szokásos potenciál feltételek mellett (vagyis a rendszerben nem folyik áram, I=0) a mintában az aktivitás lépcsőzetesen változik, ennek hatására könnyen tanulmányozható az elektrokémiai cella potenciál-idő függvénye. Nemcsak az ionszelektív indikátor elektród tulajdonságai számítanak mérvadónak, hanem a körültekintően megválasztott kísérleti körülmények, az elektrokémiai cella kialakítása, a mérési technika, illetve a cellában előforduló egyéb időfüggő potenciálforrások, mint pl.: a folytonos potenciál, a diffúzió vagy az átmeneti jel rögzítését végző elektronika. A rendszer (elektrokémiai cella +elektronika) átmeneti funkcióját az előbb felsorolt tényezők határozzák meg leginkább. \cite@article{lindner1986definition,
  title={Definition and determination of response time of ion selective electrodes},
  author={Lindner, E and Toth, Klara and Pungor, E},
  journal={Pure and Applied Chemistry},
  volume={58},
  number={3},
  pages={469--479},
  year={1986},
  publisher={De Gruyter}
}


\sectionVálaszidő meghatározása 
A gyakorlati életben kétféle módszer terjedt el, a merítéses és a befecskendezős. Pár mondatban összefoglalnám ezen módszerek lényegét. 
\subsectionMerítéses módszer
Veszünk egy ismert aio aktivitású oldatot, melyben az indikátor elektródot kondicionáljuk, ezt követően kivesszük az oldatból és letöröljük vagy lemossuk, (Az előforduló zavarok elkerülése érdekében az indikátorelektródra tapadt kondicionáló cseppeket érdemes a törlés után enyhe rázással eltávolítani.) majd t=0 időpillanatban kevert mintaoldatba (ai∞ aktivitású) merítjük, ekkor elindul az időzítő. Speciális átfolyási technikák lettek kifejlesztve annak érdekében, hogy a kísérlet folyamán bekövetkező áramkörszakadás miatt felmerült problémákat el lehessen kerülni, és emellett javuljon a reprodukálhatóság is. Tóth és munkatársai fejlesztettek ki egy berendezést, ami szintén a merítéses módszeren alapul, viszont ennél nem az indikátorelektród helyzete változik, hanem a felhasznált oldatokat mozgatjuk két, az indikátor elektród felületére merőleges mozgósugár segítségével.\cite@article{lindner1986definition,
  title={Definition and determination of response time of ion selective electrodes},
  author={Lindner, E and Toth, Klara and Pungor, E},
  journal={Pure and Applied Chemistry},
  volume={58},
  number={3},
  pages={469--479},
  year={1986},
  publisher={De Gruyter}

\subsectionBefecskendezéses módszer
Koncentrált primer ion tartalmazó oldatot (térfogata kisebb a cellatérfogatnál) fecskendeznek be egy nagy sebeséggel kevertetett tesztoldatba. A befecskendezés pillanatától számítva kezdik el mérni az időt. A cella összetételét meghatározták a befecskendezés végrehajtása előtt. \cite@article{lindner1986definition,
  title={Definition and determination of response time of ion selective electrodes},
  author={Lindner, E and Toth, Klara and Pungor, E},
  journal={Pure and Applied Chemistry},
  volume={58},
  number={3},
  pages={469--479},
  year={1986},
  publisher={De Gruyter}
}

\sectioneAz ionszelektív elektródok élettartamát befolyásoló tényezők
Elsősorban a membránösszetétel határozza meg, mivel a benne lévő ionofór, lágyító és a rögzítetlen ionok helyek kioldódhatnak a mintaoldatba, ezzel megváltozik a membránösszetétel, ennek az lesz a következménye, hogy romlik a szelektivitás vagy az elektródválasz, de akár mindkettő egyidőben is bekövetkezhet. A kioldódás mellett további tényezők csökkentik a várható élettartamot. A membránkomponensek elbomlanak vagy fizikailag sérül a membrán, ugyanakkor a belső és a mintaoldat a normálistól eltérő módon történő érintkezésének következménye, hogy rövidzárlat történik. A bomlást előidézheti, ha a vizsgált közeg savas jellegű vagy fény éri a membránt, de akár bizonyos sók jelenléte is beindíthatja a bomlást, mint pl.: a tetrafenil-borát sói. \cite@article{erdHosy2001planaris,
  title={Plan{\'a}ris fel{\'e}p{\'\i}t{\'e}s{\H{u}} ionszelekt{\'\i}v elektr{\'o}dok elm{\'e}lete {\'e}s biol{\'o}giai alkalmaz{\'a}sa},
  author={Erd{\H{o}}sy, Mikl{\'o}s},
  year={2001},
  publisher={Budapesti M{\H{u}}szaki {\'e}s Gazdas{\'a}gtudom{\'a}nyi Egyetem}
}
\sectionAz ionszelektív elektródok alkotóeleminek főbb tulajdonságai
A polimer alapú ionszelektív membránok lágyító tartalma 33% és 66% közé tehető, ennek előnye, hogy a membránkomponensek magas mozgékonyságát teszi lehetővé. Azonban, ha a lágyítótartalom 66%-ról 20%-ra csökken, akkor öt nagyságrenddel nő meg a csökkent ionmozgékonyság következtében a membrán ellenállása, viszont az ellenállás növekedését a membránkomponensek lecsökkent oldhatósága is okozhatja. A lágyítóval szembeni elvárás, hogy polaritása lineáris legyen a mérési tartományra és elsősegítse az ionpárképződést. Ahhoz, hogy a megfelelő lágyítót tudjuk megválasztani, szükséges annak ismerete, hogy milyen területen fogjuk használni az elektródot.  Az oldószerrel szemben elvárt fontos követelmény, hogy kompatibilis legyen a polimermátrixszal, vagyis lágyítóként viselkedjen, hiszen fontos a membránok szelektivitásának befolyásolása szempontjából is. Poláris jelleggel bíró lágyító kevésbé kedvez az egyszeres töltéssel rendelkező ionoknak, inkább a kétszeres töltésű ionokra fejt ki kedvező hatást. Biológiai mintáknál figyelembe kell venni, hogy a fehérjék felületi adhézióra képesek, emiatt a potenciálváltozás folytonos, illetve, hogy a lágyító az élő szövetekben gyulladásos reakciókat képes kiváltani, mert a lipofil tulajdonsága miatt képes kioldódni a membránból. Az előbb említett probléma megelőzhető, ha a lágyítót használat előtt fotopolimerizációnak vetjük alá vagy pedig nagy molekulasúlyú lágyítót használunk. \cite@article{erdHosy2001planaris,
  title={Plan{\'a}ris fel{\'e}p{\'\i}t{\'e}s{\H{u}} ionszelekt{\'\i}v elektr{\'o}dok elm{\'e}lete {\'e}s biol{\'o}giai alkalmaz{\'a}sa},
  author={Erd{\H{o}}sy, Mikl{\'o}s},
  year={2001},
  publisher={Budapesti M{\H{u}}szaki {\'e}s Gazdas{\'a}gtudom{\'a}nyi Egyetem}
}
\subsectionIonszelektív elektródok mátrixaként használatos polimer
Gyakorlatban a PVC mátrix alapú membrán terjedt el, viszont ismeretes, hogy az ionszelektív elektródok az ovostudomány területén is gyakran alkalmazott mérőeszközök, ezért olyan módosított PVC mátrixot kell megalkotni, ami optimálisan működik az élő szervezetben fenálló körülmények ellenére, mint pl: a vérszérum vagy más folyadék kevert oldatnak tekinthető és nemcsak az elsődleges, hanem zavaró ionok is jelen vannak, így az elektródnak nagy szelektivitással  kell rendelkezni, vagy hogy a membrán ne tartalmazzon erősen lipofil összetevőket, mert kioldódásukkal felbomlik a szervezetben az egyensúly és súlyos következményei lehetnek, mivel idegen anyag, ezért gyulladást is okozhat a szervezetben.  Elméleti és kísérleti úton egyaránt szerették volna megvizsgálni az analitikai paraméterek (stabilitás, reprodukálhatóság, élettartam) és potenciometriás viselkedésük alapján, hogy ha karboxil-, amin- vagy hidroxil-csoportokat kapcsoltak a PVC polimer láncaihoz, akkor a membrán milyen tulajdonságokkal fog rendelkezni, melyek befolyásolhatják az elektród alkalmazási területét. Nem szignifikáns eltéréseket tapasztaltak  a PVC-HMW (high molecular weight) membránra vonatkozóan, a korábban és a most mért adatok egyezést mutattak. Összehasonlítást végeztek annak érdekében, hogy megnézzék van-e szignifikáns különbség a beépített, mozgékony ionofórt tartalmazó és ionofórt nem tartalmazó módosított PVC-NH2 mátrix alapú membránok között. A potenciál-idő függvény görbéjének meredeksége csökkent, illetve alacsonyabb pH értékeken a kimutatási határ eltolódott, ez az ionofór nélküli membránnál volt tapasztalható, ezzel szemben az ionofór tartalmú membrán esetén a paraméterek konstans értéken álltak. Ionofórok oldatósága a PVC-NH2 és PVC-COOH mátrixokban a legnagyobb, a PVC-OH mátrixban pedig a legkisebb. Az elektródválasz lineáris tartománya a pH-érték csökkentésével, DOS(Bis(2-etilhexil)szebakát) lágyító és PVC-COOH mátrix alapú membrán használatával kiterjeszthető.\cite@article{lindner1994ion,
  title={Ion-selective membranes with low plasticizer content: electroanalytical characterization and biocompatibility studies},
  author={Lindner, E and Cosofret, VV and Ufer, S and Buck, RP and Kao, WJ and Neuman, MR and Anderson, JM},
  journal={Journal of biomedical materials research},
  volume={28},
  number={5},
  pages={591--601},
  year={1994},
  publisher={Wiley Online Library}
}

\subsectionMátrixként használt más polimerek
 A vizsgálni kívánt iontól függően lehet megválasztani a felhasználni kívánt polimer tulajdonságait és specifikussá tenni az adott ionra. Gondolok itt a méret, töltés szerinti szeparációra.  Pár mondatban ezeket az elektródfajtákat mutatom be.
Porózus membránok réteges szerkezetűek, melyek elválasztják egymástól az elektrolitokat. Diffúzió útján történik az anionok és kationok vándorlása a rétegek között. Az egyes rétegek különböző méretű pórusokkal vannak ellátva, így az ionok méretüktől függően képesek átjutni a membránrétegeken. 
Perm-szelektív membránokban mátrix található, ami rögzített ioncsoportokat tartalmaz, ezáltal a membrán permanens töltésű, ennek a koncentrációja határozza meg, hogy mekkora koncentrációtartományban képes ellentétes töltésű ionok szállítására. Ionspecifikus membránon féligáteresztő tulajdonságából adódóan csak adott ionok juthatnak át. 
Nemcsak a membránszerkezet alapján lehet az elektródokat osztályozni, hanem a felhasznált alapanyag szerint is, homogén és heterogén membránt tartalmazókra. A heterogén membrán két részből épül fel, a hordozó és a kémiailag aktív anyagból. A homogén membrán ehhez képest egyszerűbb felépítésű, mert ott a hordozóanyag egyben kémiailag aktív. Általában monomer anyag, ami egykristályos vagy amorf szerkezetű, azonban felhasználható még polikristályos szerkezetű anyag is. \cite@article{pungor1970ion,
  title={Ion-selective membrane electrodes. A review},
  author={Pungor, Ern{\"o} and T{\'o}th, Kl{\'a}ra},
  journal={Analyst},
  volume={95},
  number={1132},
  pages={625--648},
  year={1970},
  publisher={Royal Society of Chemistry}
}

Néhány mondatban foglalom össze, hogy milyen sejtek(gombák) viselkedését követtem nyomon ionszelektív elektróddal, és a felhasznált vegyületeket, melyekkel a sejteket kezeltem a mérés során.
\sectionCandida albicans
Eukarióta sejtfelépítéssel rendelkező, dimorf kórokozó gombafaj. Sarjadzó gombák közé tartozó egysejtű. Gömb vagy ovális alakú sejt, átmérője 3-6 mikrométer. Kiváló példa a mitotikus differenciálódási folyamatra, mert adott körülmények között rügyeket és csírákat alakít ki, melyek pszeudohifákat építenek fel és beindul a sejtnövekedés, viszont az utódsejtek füzérláncokat alkotva együtt maradnak, így nem teljes az elválasztódás az anyasejtektől.\cite@article{orsolya2014mannich,
  title={Mannich-ketonok antifung{\'a}lis hat{\'a}s{\'a}nak vizsg{\'a}lata {\'e}s antimikotikum-{\'e}rz{\'e}kenys{\'e}g meghat{\'a}roz{\'a}sa chip-alap{\'u} m{\'o}dszerekkel},
  author={Orsolya, Bouquet},
  year={2014}
}
 A fentebb említett tulajdonságából fakadóan képes micéliumok létrehozására, ami következtében az emberi szervezet súlyos fertőzést szenved. \cite@article{chaffin1984relationship,
  title={The relationship between yeast cell size and cell division in Candida albicans},
  author={Chaffin, W LaJean},
  journal={Canadian journal of microbiology},
  volume={30},
  number={2},
  pages={192--203},
  year={1984},
  publisher={NRC Research Press Ottawa, Canada}
}
Közismert, hogy jelen van az emberi szervezetben, mint élesztőgomba, de ameddig nem növekszik a gazdatest érzékenysége (ez alatt azt kell érteni, hogy olyan változások következnek be, melyek gyengítik az immunrendszert, így a növekedéshez megfelelő körülmények alakulnak ki), ami a legfőbb kiváltó oka a Candida albicans általi megfertőződésnek, és nem az élesztőgomba virulanciájának tulajdonítható. Leggyakrabban a genitális területeket támadja meg, de számos tanulmányban olvasható, hogy candiózisban szenvedő betegektől vett mintákból kiderült, hogy a szájüregben és az urogenitális területeken is egyaránt előfordulhat. \cite@article{odds1983candida,
  title={Candida albicans strain types from the genitalia of patients with and without Candida infection},
  author={Odds, FC and Abbott, AB and Reed, TAG and Willmott, FE},
  journal={European Journal of Obstetrics \& Gynecology and Reproductive Biology},
  volume={15},
  number={1},
  pages={37--43},
  year={1983},
  publisher={Elsevier}
}
\sectionNystatin
Streptomyces noursei nevű gombatörzs termeli ezt a polién makrolid típusú vegyületet. Szerkezetét tekintve 20-40 szénatomos makrolaktron gyűrűből és a hozzá kapcsolódó dezoxi-cukor-mikozaminból épül fel. A makrolakton gyűrűben 4-8 db konjugált kettős kötés található, melyek fontos szereppel bírnak, hiszen képesek kölcsönhatásba lépni a gombamembránt felépítő szterinekkel. Ugyanakkor fel tud venni pórusszerű szerkezetet, így a membránon átjutva okozza a sejtalkotók szivárgását, anyagcserezavarokat, legvégső esetben pedig a sejt halálát. Képes a magasabb rendű élőlények sejtjeit alkotó koleszterinhez kapcsolódni és ugyanúgy károsítani, mint egy gomba vagy hozzá hasonló alacsonyabb rendűt, ezért a gombás fertőzések kezelése során okozhat mellékhatásokat, ennek ellenére még mindig az első számú dezorganizáló vegyületnek számít, mert kórokozói viszonylag ritkák. Elsősorban gasztrointesztinális, genitális candiosis, orális fertőzések kezelésére használják.\cite@article{fjaervik2005biosynthesis,
  title={Biosynthesis of the polyene macrolide antibiotic nystatin in Streptomyces noursei},
  author={Fj{\ae}rvik, Espen and Zotchev, Sergey B},
  journal={Applied microbiology and biotechnology},
  volume={67},
  number={4},
  pages={436--443},
  year={2005},
  publisher={Springer}
}

\sectionPropolis
Gyantás, gumiszerű állaggal rendelkezik, ezért aromás ragasztókhoz hasonlítják. Korától és származási helyétől függően változik a színe, a sárga és zöld színtől kezdve egészen asötétbarnáig. A gumiszerű ragadós jelleg akkor mutatkozik, mikor meleg az anyag, éppen ezért nehéz az emberi bőrről eltávolítani, a másik ok pedig, hogy kölcsönhatásba lép a bőrben található olajokkal és a bőr fehérjéivel. Viszont mikor hideg, kemény és törékeny tulajdonságú. Eddigi kutatómunkák alapján a propolis összetételét illetően arra jutottak, hogy nagy részben flavonoidokat tartalmaz, melyek a növényvilágban is megtalálhatóak. Különböző helyről származó propolis mintákban flavonokat (apigenin, galangin, pinocembrin, kvercetin), egyszerű aromás vegyületeket (szalicil-, gallusz-, ferul-, koffein-, p-kumarin-sav) benzaldehid származékokat, fahéjsav észtereit(vanillin) normál szénláncú karbonvegyületeket(alkoholok, észterek, savak), cukrokat, aminosavakat,\cite@article{woisky1998analysis,
  title={Analysis of propolis: some parameters and procedures for chemical quality control},
  author={Woisky, Ricardo G and Salatino, Antonio},
  journal={Journal of apicultural research},
  volume={37},
  number={2},
  pages={99--105},
  year={1998},
  publisher={Taylor \& Francis}
}szennyező anyagokat (ólom 9ppb-nél kisebb koncentrációban, színanyagok) izoláltak, melyek hasonlóságot mutattak a növényekben lévőkkel, amelyek virágporát a méhek begyűjtötték. A méhek nyálában található enzim jelenlétében a flavonok egy része transzformálódik, ez a folyamat a gyűjtés során játszódik le. \cite@article{burdock1998review,
  title={Review of the biological properties and toxicity of bee propolis (propolis)},
  author={Burdock, GA},
  journal={Food and Chemical toxicology},
  volume={36},
  number={4},
  pages={347--363},
  year={1998},
  publisher={Elsevier}
} Korábbi tanulmányok alátámasztották antivirális, antibakteriális és antifungális hatását. Antifungális szereknek növeli a hatékonyságát. A legtöbb kutatásban arra jutottak, hogy parazitapusztító tulajdonságát a benne található flavonok közül a kvercetin, és a koffeinsav fenil-észtere okozzák. \cite@article{marcucci1995propolis,
  title={Propolis: chemical composition, biological properties and therapeutic activity},
  author={Marcucci, Maria Cristina},
  journal={Apidologie},
  volume={26},
  number={2},
  pages={83--99},
  year={1995},
  publisher={EDP Sciences}
}

\sectionVégül, de nem utolsósorban szeretnék elmítést tenni a pásztázó elektrokémiai mikroszkópiáról(SECM), hiszen részben kapcsolódik az általam feldolgozott témához és az elmúlt években széles körben alkalmazták különféle elektrokémiai rendszerekben, különösen nagy térbeli felbontású, kis térfogatú mintában elektrokémiai mérések során. A többi típusú pásztázó szonda mikroszkóppal szemben a SECM egyik előnye, hogy információt tud nyújtani a  vizsgálni kívánt szubsztrát kémiai természetéről és az őt körülvevő környezetről, ezeket a tényezőket a felület topográfiai képeiről lehet leolvasni. Általában SECM-et amperometrikus módban használták a szubsztrátok szigetelő és vezető régióinak megkülönböztetésére, a vezetőképességgel rendelkező szubsztrát felületi aktivitásának letérképezésére, immobilizált redox-aktív enzimek kötőhelyeinek meghatározására adott felületen és elektroaktív egyed lokális koncentrációjának feltérképezésére. Sajnálatosan nem minden kémiai rendszer esetén lehet alkalmazni amperometriás technikákat, mert a vizsgálni kívánt egyed nem elektroaktív, mint pl.: alkáli- és alkáliföldfémionok vizes közegben nehezen határozhatók meg voltametriás módszerek segítségével, ugyanakkor a meghatározást korlátozhatja az interferencia, ezt háttéráram okozza. Hatókörének kiterjesztése érdekében szükség volt új típusú érzékelők, mint pásztázó szondák, potenciometrikus érzékelők, ion-szelektív mikroelektródák (ISME-k) kifejlesztésére. Különböző ionok lokális aktivitásának potenciometrikus meghatározására fejlesztették ki az ISME-ket.\cite@article{wei1995scanning,
  title={Scanning electrochemical microscopy. 28. Ion-selective neutral carrier-based microelectrode potentiometry},
  author={Wei, Chang and Bard, Allen J and Nagy, Geza and Toth, Klara},
  journal={Analytical Chemistry},
  volume={67},
  number={8},
  pages={1346--1356},
  year={1995},
  publisher={ACS Publications}
}
 Korábbi fejlesztéseknek köszönhetően nagy térbeli felbontású képeket lehet  a segítségükkel készíteni. Klusmann és Schultze pH-érzékelny mikropipetta hegyet használt kísérleteikben, hogy összehasonlítsák az általuk számított pH-profilokat más mérési eredményekkel. Hagyományos mikropipetta elektród készítésének két kritikus lépése van, amelyek meghatározzák a sikerességi arányt. Az egyik a koktél befecskendezése a kapillárisba, a másik pedig, hogy a belső töltőoldatot hogyan áramoltatjuk az elektródtestben. Annál nehezebb ezeket a lépéseket végrehajtani, minél kisebb a hegy mérete, így nő a meghibásodás esélye. Ha a koktél és a vizes oldat közé légbuborék kerül nem megfelelően fog működni az elektród, mert a buborék szigetelő anyagként fog viselkedni. A hagyományos mikropipetta elektród elveszítheti mérőfunkcióját, mikor a vizes oldat beszivárog a koktélon keresztül az üvegkapillárisba, a kappiláris fala átnedvesedik, ez azonban lassabban megy végbe, ha a koktél a vizes oldattal csak a kapilláris egyik oldalán érintkezik. \cite@article{gyetvai2007solid,
  title={Solid contact micropipette ion selective electrode for potentiometric SECM},
  author={Gyetvai, Gergely and Sundblom, S{\"o}ren and Nagy, L{\'\i}via and Ivaska, Ari and Nagy, G{\'e}za},
  journal={Electroanalysis: An International Journal Devoted to Fundamental and Practical Aspects of Electroanalysis},
  volume={19},
  number={10},
  pages={1116--1122},
  year={2007},
  publisher={Wiley Online Library}
}
Rutinszerűen alkalmazzák az ISME-t élő sejtek intra-és extracelluláris ionaktivitásának mérésére, és számos biológiailag fontos ion kimutatható potenciometriás módszerekkel. Egy adott ion lokális koncentrációjának kvantitatív mérésére a potenciometrikus SECM jobb, mint az amperometrikus detektálás, mivel a potenciometrikus csúcselektród nem okoz jelentős változást a  vizsgált egyed koncentrációjában. Ráadásul a potenciometrikus mérés lehetővé teszi a helyi ionaktivitás közvetlen kiszámítását a kalibrációs görbéből, míg az amperometrikus SECM-ben a csúcs jelét a hordozó felületi topográfiája is befolyásolja, mert a koncentrációt a visszacsatoló áram határozza meg. Potenciometrikus SECM általában jobb szelektivitást biztosít, mint amperometrikus megfelelője. Antimon mikrolemez elektród használható kémiai rendszerekben a lokális pH-változások letérképezésére. Főként fémalapú ISME-ket alkalmaztak a potenciometriás SECM területén, mint pl. Ag, Sb. Ezek előnye, hogy gyorsan reagálnak, könnyen legyárthatók(hasonlóan a hagyományos UME-hoz), valamint az elektród csúcsa és a érintett felület közti távolságon áthaladó áramot képes detektálni. A fémalapú ISME-k csak néhány különböző ion (pl. Ag+, Cl-, H+) kimutatására használhatók, és a mérőcsúcsuk elkészítéséhez használt módszerek nem ültethetők át más ISME-okra, mert a legtöbb ISME-t nem azonos módon kell megépíteni. Fejlődésük egyértelműen a folyékony membrán mikroelektródák irányába mutat, amely a membránban található rendkívül szelektív semleges hordozók használatán alapul. A folyékony membrán, illetve üvegkapilláris alapú  mikropipetta ISME-ket széles körben használják élettudományi kísérletekben. A legtöbb elektród esetén az a probléma merül fel, hogy nehéz behatárolni a mérőcsúcs abszolút helyzetét. Fontos szempontnak számít az is, hogy a mérőcsúcs és a vizsgálandó egyed közötti abszolút távolságot úgy lehessen meghatározni, hogy mérés során egyik se szenvedjen sérülést, ugyanakkor a felszín közeli koncentrációprofilokból menniységi információt tudjunk kinyerni. Adott körülmények között az optikai mikroszkóp megfelelőnek bizonyul, viszont nem lehet mikron tartományban történő mérésre használni.\cite@article{wei1995scanning,
  title={Scanning electrochemical microscopy. 28. Ion-selective neutral carrier-based microelectrode potentiometry},
  author={Wei, Chang and Bard, Allen J and Nagy, Geza and Toth, Klara},
  journal={Analytical Chemistry},
  volume={67},
  number={8},
  pages={1346--1356},
  year={1995},
  publisher={ACS Publications}
}

\pagestyle{headings}

 
\pagestyle{headings}

